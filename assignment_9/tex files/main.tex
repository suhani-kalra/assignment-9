\documentclass{report}
\usepackage[utf8]{inputenc}
\usepackage{enumitem}
\title{Assignment 9 (GATE, EC2017,45)}
\author{Suhani Kalra }
\date{16 December 2020}


\usepackage{circuitikz}
\usepackage{array}
\begin{document}
\LARGE



\maketitle


\section{QUESTION}

\begin{figure}[htp]
\centering
\includegraphics[width=17cm]{image.pdf}
\caption{GATE, EC2017,45}
\label{fig:ok}
\end{figure}

A programmable logic array (PLA) is shown in the figure.The Boolean function F implemented is

 \begin{enumerate}
 \item P'Q'R+PQ'R+P'QR
\item ( P'+Q'+R)(P'+Q+R)(P+Q'+R')
\item P'QR+PQ'R+PQR
\item ( P+Q'+R)(P'+Q+R)(P+Q'+R')
  \end{enumerate}




\newpage
\section{SOLUTION}
For the question 45 of GATE EC 2017 exam(Figure 1, Section 0.1),
\newline
we had to deduce the expression for the final Output F, given in Figure 1 (section0.1) for this, we can say that, Figure 1 could be reduced to the following equation
\begin{equation}
   \textcolor{blue}{F = P'Q'R+PQ'R+P'QR}
\end{equation}
 The following report aims at providing a detailed derivation of the equation1 using,Truth Table and Logical Circuit Diagram.
   


\newpage
\section{CIRCUIT DIAGRAM}

The given question (figure 1, section 0.1 ) can also be represented by the following circuit diagram. This circuit diagram helps us to deduce the above mentioned equation1 (section0.2).
\begin{figure}[h]\centering
\begin{circuitikz}

\ctikzset{tripoles/american and port/height=1.1, number inputs=3 };
\ctikzset{tripoles/american or port/height=1};
\draw

(2,0.7)node[american or port, number inputs=3, anchor=in 1](OR1) {} 
(-1.4,-1) node[american and port, number inputs=3, anchor=in 1](a){PQ'R}
(-1.4,0.7) node[american and port, number inputs=3, anchor=in 1](C){P'Q'R}
(-1.4,2.5) node[american and port, number inputs=3, anchor=in 1](b){P'QR}

(C.in 1) node[anchor=east] {P}
(C.in 2) node[anchor=east] {Q}
(C.in 3) node[anchor=east] {R}
(b.in 1) node[anchor=east] {P}
(b.in 2) node[anchor=east] {Q}
(b.in 3) node[anchor=east] {R}
(a.in 1) node[anchor=east] {P}
(a.in 2) node[anchor=east] {Q}
(a.in 3) node[anchor=east] {R}

(OR1.out) node[anchor=west] {$F$}

(a.out)--(OR1.in 3)
(b.out)--(OR1.in 1)
(C.out)--(OR1.in 2)
;
\node at (b.bin 1) [ocirc, left]{} ;
\node at (C.bin 2) [ocirc, left]{} ;
\node at (C.bin 1) [ocirc, left]{} ;
\node at (a.bin 2) [ocirc, left]{} ;

\end{circuitikz}
\end{figure}

\newpage

\section{TRUTH TABLE}
TRUTH TABLE-
The given question (figure 1, section 0.1 ) can also be represented by the eforementioned Truth Table. This Truth table helps us to deduce the above mentioned equation1 (section0.2).

\begin{table}[]\centering
\begin{tabular}{|l|l|l|l|}
 \hline
 
P & Q & R & F  \\ \hline
0 & 0 & 0     & 0        \\
0 & 0 & 1     & 1       \\
0 & 1 & 0     & 0        \\ 
0 & 1 & 1     & 1        \\
1 & 0 & 0     & 0         \\
1 & 0 & 1     & 1         \\
1 & 1 & 0     & 0         \\
1 & 1 & 1     & 0         \\ \hline

\end{tabular}
\end{table}


\end{document}
